\documentclass{beamer}

%Packages
\usepackage{graphicx}
\usepackage{hyperref}

\graphicspath{{./figs/}}

%Presentation info
\title[]{Using Code and Scripts to Navigate an NES Game}
\subtitle{A Freeware Homebrew NES Game Because Legality}
\author[]{}
\institute{Coding Dojo}
\date{}

\begin{document}

\begin{frame}
	\titlepage
\end{frame}

\begin{frame}{Important Preface}
	\begin{itemize}
		\item There is a good chance this may be a bit too involved for one evening
		\item If this goes too poorly, we can migrate to a different location to discuss what we have learned
	\end{itemize}
\end{frame}

\begin{frame}{Introduction}
	\begin{itemize}
		\item As many of us probably know, computers are good for more than just math and kitty cats and completely dominating human players in video games.
		\item They are also good for controlling the movement of expensive and/or expendable devices
		\begin{itemize}
			\item Like cars
			\item Or Drones
			\item Or really expensive lab equipment
			\item And humanoid robots with a heart of gold that resemble Jude Law and spend a lot of time around small children
		\end{itemize}
		\item Able to react much more quickly than a human
		\item Able to much more consistently reproduce a sequence of actions
	\end{itemize}
\end{frame}

\begin{frame}{}
	\begin{itemize}
		\item So we should do that
		\item Except the logistics of controlling devices is complicated
		\item Fortunately, video games
	\end{itemize}
\end{frame}


\begin{frame}{}
	\begin{itemize}
		\item People with ridiculously good reaction times and dexterity like to play video games really quickly
		\item Based around memorizing the entire game and performing many ``frame perfect'' tricks (actions must occur within a one sixtieth of a second window)
		\item Others, with worse reaction time but more free time, use special tools to record a perfect run
		\item Allows for a sequence of button presses to be recorded and played back, allowing for each action to be the best possible execution
		\item But they prefer to record human input, whereas we are going to just script things
	\end{itemize}
\end{frame}


\begin{frame}{FCEUX}
	\begin{itemize}
		\item For this dojo, we will use FCEUX
		\item An NES emulator with the ability to use a Lua script as input
		\item Selected because it allows scriptable input and supports Windows and Linux (and probably Mac?)
	\end{itemize}
\end{frame}

\begin{frame}{Super Bat Puncher}
	\begin{itemize}
		\item We will use the Lua script interface to FCEUX to try to complete the Super Bat Puncher demo
		\item Selected because it was the least horrible looking freeware NES rom that probably doesn't use proprietary NES sprites
		\item The one I really wanted to do costs about 30 bucks, so it isn't going to happen
	\end{itemize}
\end{frame}

\begin{frame}{The Goal}
	\begin{itemize}
		\item The goal of this demo is to make as much progress as you can in the demo before we reach critical mass of giving up
		\item A useful starter script can be found in the repository
		\item FCEUX: \url{http://www.fceux.com/web/home.html} 
		\item FCEUX Lua Documentation \url{http://www.fceux.com/web/help/fceux.html?LuaFunctionsList.html}
		\item Super Bat Puncher: \url{http://morphcat.de/superbatpuncher/}
	\end{itemize}
\end{frame}


\end{document}
